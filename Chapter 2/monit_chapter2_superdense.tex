\paragraph{2.69} \textbf{Bell Basis}%%%%%%%%%%%%%%%%%%%%%%%%%%%%%%%%%%%%%%%%%%%%
\\

We know the bell states are:

$$|\psi_1\rangle = \frac{\ket{00}+\ket{11}}{\sqrt{2}} = \frac{1}{\sqrt{2}}\begin{bmatrix}
    1 \\ 0 \\ 0 \\ 1
\end{bmatrix}$$

$$|\psi_2\rangle = \frac{\ket{00}-\ket{11}}{\sqrt{2}} = \frac{1}{\sqrt{2}}\begin{bmatrix}
    1 \\ 0 \\ 0 \\ -1
\end{bmatrix}$$

$$|\psi_3\rangle = \frac{\ket{01}+\ket{10}}{\sqrt{2}} = \frac{1}{\sqrt{2}}\begin{bmatrix}
    0 \\ 1 \\ 1 \\ 0
\end{bmatrix}$$

$$|\psi_4\rangle = \frac{\ket{01}-\ket{10}}{\sqrt{2}} = \frac{1}{\sqrt{2}}\begin{bmatrix}
    0 \\ 1 \\ -1 \\ 0
\end{bmatrix}$$

To form a basis, it must be linearly independent, Hence:

$$ a_1 \ket{\psi_1} + a_2 \ket{\psi_2} + a_3 \ket{\psi_3} + a_4 \ket{\psi_4} = 0$$

$$ \implies \frac{1}{\sqrt{2}} \begin{bmatrix}
    a_1 + a_2 \\ a_3 + a_4 \\ a_3 - a_4 \\ a_1 - a_2
\end{bmatrix} = 0$$

$$ \implies a_1 = a_2 = a_3 = a_4 = 0$$

Thus $\{\ket{\psi_i}\}$ is a linearly independent basis.

Moreover $||\ket{\psi_i}|| = 1$ and $\braket{\psi_i|\psi_j} = \delta_{ij}$ for $i, j = 1,2,3,4$. Thus it forms an orthonormal basis.

\paragraph{2.70} \textbf{Bell States}%%%%%%%%%%%%%%%%%%%%%%%%%%%%%%%%%%%%%%%%%%%%
\\
For any Bell state we have $ \braket{\psi_i| E \otimes I|\psi_i} = \frac{1}{2}(\braket{0|E|0} + \braket{1|E|1})$

Suppose Eve measures the qubit Alice sent by measurement operators $M_m$. The probability that Eve get result $m$ is $p_i(m) = \braket{\psi_i|M_m^{\dagger} M_m \otimes I | \psi_i}$ Since $M_m^{\dagger}M_m$ is positive , $p_i(m)$ are same values for all $\ket{\psi_i}$ . Thus Eve can't distinguish Bell states.


\paragraph{2.71} \textbf{Mixed or Pure}%%%%%%%%%%%%%%%%%%%%%%%%%%%%%%%%%%%%%%%%%%%%
\\
 We know from spectral decomposition
 $$ \rho = \sum_i p_i  |\psi_i\rangle \langle\psi_i|, \ \ p_i \ge 0 , \ \ \sum_i p_i = 1$$
$$ \rho^2 = \sum_{i,j} p_i p_j |i\rangle \langle i|  |j\rangle \langle j|$$
$$ = \sum_{i,j} p_i p_j  |i\rangle \langle i| \delta_{ij}$$
$$ = \sum_i p_i^2  |i\rangle \langle i|$$

$$ Tr(\rho^2) = Tr(\sum_i p_i^2  |i\rangle \langle i|) = \sum_i p_i^2 Tr( |i\rangle \langle i|) = \sum_i p_i^2  |i\rangle \langle i| = \sum_i p_i^2 \ge \sum_i p_i = 1$$

Suppose $Tr(\rho^2) = 1$. Then $\sum_i p_i^2 = 1$. Since $p_i^2 < p_i$ for $0 < p_i < 1$, only single $p_i $ should be $1$ and otherwise have to vanish. Therefore $\rho =  |\psi_i\rangle \langle \psi_i|$. It is a pure state.

Conversely if $\rho$ is pure, then $\rho =  |\psi_i\rangle \langle \psi_i|$
$$ Tr(\rho^2) = Tr( |\psi_i\rangle \langle \psi_i| |\psi_i\rangle \langle \psi_i|) = Tr( |\psi_i\rangle \langle \psi_i|) =  |\psi_i\rangle \langle \psi_i| = 1$$


\paragraph{2.72} \textbf{Bloch Sphere}%%%%%%%%%%%%%%%%%%%%%%%%%%%%%%%%%%%%%%%%%%%%
\\

(1) Since density matrix is Hermitian, matrix representation is
$\rho = \begin{bmatrix}
    a & b \\ b^* & d
\end{bmatrix}$,
$a, d \in \mathbb{R}$ and $b \in \mathbb{C}$ w.r.t. standard basis.
Because $\rho$ is density matrix, $\Tr(\rho) = a+d = 1$.

Define $a = (1+r_3)/2$, $d = (1-r_3)/2$ and $b = (r_1 - \iota r_2)/2$, $(r_i \in \mathbb{R})$.

In this case,
$$
    \rho = \begin{bmatrix}
        a & b \\ b^* & d
    \end{bmatrix}
    =
    \frac{1}{2} \begin{bmatrix}
        1+r_3 & r_1 - ir_2 \\
        r_1 + ir_2 & 1 - r_3
    \end{bmatrix}
    =
    \frac{1}{2} (I + \vec{r} \cdot \vec{\sigma}).
$$
Thus for arbitrary density matrix $\rho$ can be written as $\rho = \frac{1}{2} (I + \vec{r} \cdot \vec{\sigma})$.

Next, we derive the condition that $\rho$ is positive.

If $\rho$ is positive, all eigenvalues of $\rho$ should be non-negative.
$$
    \det (\rho - \lambda I) &= (a-  \lambda) (b - \lambda) - |b|^2 = \lambda^2 - (a+d)\lambda + ad - |b^2| = 0$$
    $$
    \lambda = \frac{(a+d) \pm \sqrt{(a+d)^2 - 4 (ad - |b|^2)}}{2}$$
    $$
        = \frac{1 \pm \sqrt{1 - 4 \left(\frac{1 - r_3^2}{4} - \frac{r_1^2 + r_2^2}{4} \right)}}{2}$$
        $$
        = \frac{1 \pm \sqrt{1 - (1 - r_1^2 - r_2^2 - r_3^2)}}{2}\\
        = \frac{1 \pm \sqrt{|\vec{r}|^2}}{2}\\
        = \frac{1 \pm |\vec{r}|}{2}
$$

Since $\rho$ is positive, $\frac{1 - |\vec{r}|}{2} \geq 0 \rightarrow |\vec{r}| \leq 1$.

Therefore an arbitrary density matrix for a mixed state qubit is written as $\rho = \frac{1}{2} (I + \vec{r} \cdot \vec{\sigma})$.

\vspace{5mm}
(2)

$\rho = I / 2 \rightarrow \vec{r}  = 0$. Thus  $\rho = I / 2$ corresponds to the origin of Bloch sphere.

\vspace{5mm}
(3)

$$
    \rho^2 = \frac{1}{2} (I + \vec{r} \cdot \vec{\sigma})~ \frac{1}{2} (I + \vec{r} \cdot \vec{\sigma})$$
    $$
        = \frac{1}{4} \left[ I + 2 \vec{r}\cdot \vec{\sigma} + \sum_{j,k}r_j r_k \left(\delta_{jk} I + i \sum_{l=1}^3 \epsilon_{jkl}\sigma_l \right)  \right]$$
        $$
        = \frac{1}{4} \left(I + 2 \vec{r}\cdot \vec{\sigma} + |\vec{r}|^2 I \right)$$
        $$\Tr (\rho^2) = \frac{1}{4} (2 + 2|\vec{r}|^2)$$

If $\rho$ is pure, then $\Tr (\rho^2) = 1$.
$$   1 =  \Tr (\rho^2) = \frac{1}{4} (2 + 2|\vec{r}|^2)$$
$$   \therefore |\vec{r}| = 1.$$

Conversely, if $|\vec{r}| = 1$, then $\Tr (\rho^2) = \frac{1}{4} (2 + 2|\vec{r}|^2) = 1$. Therefore $\rho$ is pure.


\paragraph{2.73} \textbf{}%%%%%%%%%%%%%%%%%%%%%%%%%%%%%%%%%%%%%%%%%%%%
\\

\begin{screen}
    \Textbf{Theorem 2.6}
%
    \begin{align*}
        \rho = \sum_i p_i \kb{\psi_i}
            = \sum_i \kb{\tilde{\psi_i}}
            = \sum_j \kb{\tilde{\varphi}_j}
            = \sum_j q_j \kb{\varphi_j}
                ~~ \Leftrightarrow ~~
            \ket{\tilde{\psi}_i} = \sum_j u_{ij} \ket{\tilde{\varphi}_j}
    \end{align*}
    where $u$ is unitary.

	The-transformation in theorem 2.6, $\ket{\tilde{\psi}_i} = \sum_j u_{ij} \ket{\tilde{\varphi}_j}$, corresponds to
	\begin{align*}
	    \left[ \ket{\tilde{\psi}_1} \cdots \ket{\tilde{\psi}_k} \right] = \Big[ \ket{\tilde{\varphi}_1} \cdots \ket{\tilde{\varphi}_k} \Big] U^T
	\end{align*}
	where $k = \mathrm{rank} (\mathcal{\rho})$.
    \begin{align}
        \sum_i \kb{\tilde{\psi_i}} &= \left[ \ket{\tilde{\psi}_1} \cdots \ket{\tilde{\psi}_k} \right]
            \begin{bmatrix}
                \bra{\tilde{\psi_1}}\\
                \vdots\\
                \bra{\tilde{\psi_k}}
            \end{bmatrix}\\
        &= \Big[ \ket{\tilde{\varphi}_1} \cdots \ket{\tilde{\varphi}_k} \Big] U^T
            U^* \begin{bmatrix}
                    \bra{\tilde{\varphi}_1}\\
                    \vdots\\
                    \bra{\tilde{\varphi}_k}
            \end{bmatrix}\\
        &= \Big[ \ket{\tilde{\varphi}_1} \cdots \ket{\tilde{\varphi}_k} \Big]
             \begin{bmatrix}
                \bra{\tilde{\varphi}_1}\\
                \vdots\\
                \bra{\tilde{\varphi}_k}
            \end{bmatrix}\\
        &= \sum_j \kb{\tilde{\varphi}_j}.
    \end{align}
\end{screen}

From spectral theorem, density matrix $\rho$ is decomposed as $\rho = \sum_{k=1}^{d} \lambda_k \kb{k}$ where $d = \dim \mathcal{H}$.
Without loss of generality, we can assume $p_k > 0$ for $k = 1 \cdots , l$ where $l = \mathrm{rank} (\rho)$ and $p_k = 0$ for $k = l+1, \cdots, d$.
Thus $\rho = \sum_{k=1}^{l} p_k \kb{k} = \sum_{k=1}^{l} \kb{\tilde{k}}$, where $\ket{\tilde{k}} = \sqrt{\lambda_k} \ket{k}$.

Suppose $\ket{\psi_i}$ is a state in support $\rho$. Then
\begin{align*}
	\ket{\psi_i} = \sum_{k=1}^l c_{ik} \ket{k}, ~~ \sum_k |c_{ik}|^2 = 1.
\end{align*}

Define $\displaystyle p_i = \frac{1}{\sum_k \frac{|c_{ik}|^2}{\lambda_k} }$ and $\displaystyle u_{ik} = \frac{\sqrt{p_i} c_{ik}}{\sqrt{\lambda_k}}$.

Now
\begin{align*}
	\sum_k |u_{ik}|^2 = \sum_k \frac{p_i | c_{ik} |^2 }{\lambda_k} = p_i \sum_k \frac{| c_{ik} |^2 }{\lambda_k} = 1.
\end{align*}

Next prepare an unitary operator
\footnote{By Gram-Schmidt procedure construct an orthonormal basis $\{\boldsymbol{u}_j\}$ (row vector) with $\boldsymbol{u}_i = [u_{i1} \cdots u_{ik} \cdots u_{il}]$. Then define unitary $U = \begin{bmatrix}
    \boldsymbol{u}_1 \\ 
    \vdots \\ 
    \boldsymbol{u}_i \\ 
    \vdots \\ 
    \boldsymbol{u}_l
    \end{bmatrix}$.}
such that $i$th row of $U$ is $[u_{i1} \cdots u_{ik} \cdots u_{il}]$.
Then we can define another ensemble such that
\begin{align*}
	\Big[  \ket{\tilde{\psi}_1} \cdots  \ket{\tilde{\psi}_i} \cdots \ket{\tilde{\psi}_l}\Big] = \Big[ \ket{\tilde{k}_1} \cdots \ket{\tilde{k}_l} \Big] U^T
\end{align*}
where $\ket{\tilde{\psi_i}} = \sqrt{p_i} \ket{\psi_i}$.
From theorem 2.6,
\begin{align*}
	\rho = \sum_k \kb{\tilde{k}} = \sum_k \kb{\tilde{\psi}_k}.
\end{align*}

Therefore we can obtain a minimal ensemble for $\rho$ that contains $\ket{\psi_i}$.

Moreover since $\rho^{-1} = \sum_k \frac{1}{\lambda_k} \kb{k}$,
\begin{align*}
	\braket{\psi_i | \rho^{-1} | \psi_i} = \sum_k \frac{1}{\lambda_k} \braket{\psi_i | k} \hspace{-1mm} \braket{k | \psi_i} = \sum_k \frac{|c_{ik}|^2}{\lambda_k} = \frac{1}{p_i}.
\end{align*}

Hence, $ \frac{1}{\braket{\psi_i | \rho^{-1} | \psi_i}} = p_i $.


\paragraph{2.74} \textbf{}%%%%%%%%%%%%%%%%%%%%%%%%%%%%%%%%%%%%%%%%%%%%
\\

We know
$$ \rho_{AB} = |a\rangle \langle a|_A \otimes |b\rangle \langle b|_B$$

$$\rho_A = Tr_B \rho_{AB} = |a\rangle \langle a| Tr(|b\rangle \langle b|) = |a\rangle \langle a|$$

$$ Tr(\rho_{A}^2) = 1$$

Thus $\rho_{A}$ is pure.

