
\paragraph{2.8} %%%%%%%%%%%%%%%%%%%%%%%%%%%%%%%%%%%%%%%%%%%%



\paragraph{2.9} %%%%%%%%%%%%%%%%%%%%%%%%%%%%%%%%%%%%%%%%%%%%

The identity Pauli matrix is given by,

\begin{align}
  \label{eq:2.9.1}
  \sigma_0 \equiv I \equiv
  \begin{bmatrix}
    1 & 0 \\
    0 & 1
  \end{bmatrix}
\end{align}

And must fulfill that $\hat{I}\ket{1} = \ket{1}$ and $\hat{I}\ket{0} = \ket{0}$

Then we can express the operator $\hat{I}$ as follows;

\begin{align}
  \label{eq:2.9.2}
  \Rightarrow\hat{I}=\ket{0}\bra{0}+\ket{1}\bra{1}
\end{align}

Now let's see for the remaining Pauli operators. In general, when we have an operator represented as matrix:

\begin{align}
  \label{eq:2.9.3}
  A =
  \begin{pNiceMatrix}[first-row, last-col=3]
    \bra{0} & \bra{1}          \\
    \alpha  & \beta  & \ket{0} \\
    \gamma  & \delta & \ket{1}
  \end{pNiceMatrix}
\end{align}

\begin{align}
  \label{eq:2.9.4}
  \Rightarrow \; A = \alpha\ket{0}\bra{0}+\beta\ket{0}\bra{1}+\gamma\ket{1}\bra{0}+\delta\ket{1}\bra{1}
\end{align}

Therefore,

\begin{align}
  \label{eq:2.9.5}
  \sigma_1 \equiv \sigma_x \equiv X \equiv
  \begin{bmatrix}
    0 & 1 \\
    1 & 0
  \end{bmatrix}
  \; \; \Rightarrow \; \hat{X}=\ket{0}\bra{1}+\ket{1}\bra{0}
\end{align}

\begin{align}
  \label{eq:2.9.6}
  \sigma_2 \equiv \sigma_y \equiv Y \equiv
  \begin{bmatrix}
    0 & -i \\
    i & 0
  \end{bmatrix}
  \; \; \Rightarrow \; \hat{Y}=i\ket{1}\bra{0}-i\ket{0}\bra{1}
\end{align}

\begin{align}
  \label{eq:2.9.7}
  \sigma_3 \equiv \sigma_z \equiv Z \equiv
  \begin{bmatrix}
    1 & 0 \\
    0 & -1
  \end{bmatrix}
  \; \; \Rightarrow \; \hat{Z}=\ket{0}\bra{0}-\ket{1}\bra{1}
\end{align}



\paragraph{2.10} %%%%%%%%%%%%%%%%%%%%%%%%%%%%%%%%%%%%%%%%%%%%

We are looking for the matrix representation of the operator $\ket{v_j}\bra{v_k}$, then,

\begin{align}
  \label{eq:2.10.1}
  \ket{v_j}\bra{v_k}&=I_v\ket{v_j}\bra{v_k}I_v \\
  &=\left(\sum_p\ket{v_p}\bra{v_p}\right)\ket{v_j}\bra{v_k}\left(\sum_q\ket{v_q}\bra{v_q}\right)\\
  &=\sum_{p,q}\ket{v_p}\braket{v_p|v_j}\braket{v_k|v_q}\bra{v_q}\\
  &=\sum_{p,q}\ket{v_p}\delta_{p,j}\delta_{k,q}\bra{v_q}\\
  &=\sum_{p,q}\underbrace{\delta_{p,j}\delta_{k,q}}_{A_{pq}}\ket{v_p}\bra{v_q}\\
\end{align}

Where the array element $A_{pq}$, with column $q$ and row $p$,

\begin{align}
  \label{eq:2.10.2}
  A_{pq} = \left(\ket{v_j}\bra{v_k}\right)_{pq}=\delta_{p,j}\delta_{k,q}
\end{align}


\paragraph{2.11} %%%%%%%%%%%%%%%%%%%%%%%%%%%%%%%%%%%%%%%%%%%%

Pauli matrix $\sigma_1=\sigma_x$

\begin{align}
  \label{eq:2.11.1}
  X=
  \begin{bmatrix}
    0 & 1 \\
    1 & 0
  \end{bmatrix}
\end{align}

\begin{align}
  \label{eq:2.11.2}
  \Rightarrow 0 = c(\lambda_x)&=|X-\lambda_x I|
  =
  \begin{vmatrix}
    -\lambda_x & 1 \\
    1 & -\lambda_x
  \end{vmatrix}
  =\lambda_x^2-1
\end{align}

\begin{align}
  \label{eq:2.11.3}
  \therefore \lambda_x=\pm 1
\end{align}

If $\lambda_{x,1}=1$, its eigenvector $\ket{v_{x,1}}$ is:

\begin{align}
  \label{eq:2.11.4}
  \Rightarrow X-\lambda_{x,1} I
  =
  \begin{bmatrix}
    -1 & 1 \\
    1 & -1
  \end{bmatrix} 
\end{align}

\begin{align}
  \label{eq:2.11.5}
  \Rightarrow
  \begin{bmatrix}
    -1 & 1 \\
    1 & -1
  \end{bmatrix}
  \begin{bmatrix}
    a_{x,1} \\
    a_{x,2}
  \end{bmatrix}
  =
  \begin{bmatrix}
    0 \\
    0
  \end{bmatrix}
\end{align}

\begin{align}
  \label{eq:2.11.6}
  \Rightarrow \ket{v_{x,1}} = \frac{1}{\sqrt{2}}
  \begin{bmatrix}
    1 \\
    1
  \end{bmatrix} 
\end{align}

If $\lambda_{x,2}=-1$, its eigenvector $\ket{v_{x,2}}$ is:

\begin{align}
  \label{eq:2.11.7}
  \Rightarrow X-\lambda_{x,2} I
  =
  \begin{bmatrix}
    1 & 1 \\
    1 & 1
  \end{bmatrix} 
\end{align}

\begin{align}
  \label{eq:2.11.8}
  \Rightarrow
  \begin{bmatrix}
    1 & 1 \\
    1 & 1
  \end{bmatrix}
  \begin{bmatrix}
    b_{x,1} \\
    b_{x,2}
  \end{bmatrix}
  =
  \begin{bmatrix}
    0 \\
    0
  \end{bmatrix}
\end{align}

\begin{align}
  \label{eq:2.11.9}
  \Rightarrow \ket{v_{x,2}} = \frac{1}{\sqrt{2}}
  \begin{bmatrix}
    -1 \\
    1
  \end{bmatrix} 
\end{align}

Note that the matrix representation of this in the basis of the eigenvectors for a particular operator, is just a diagonal matrix with entries equal to the eigenvalues:

\begin{align}
  \label{eq:2.11.10}
  \Rightarrow X_{diag} =
  \begin{bmatrix}
    1 & 0 \\
    0 & -1
  \end{bmatrix} 
\end{align}


------------------------------------

Pauli matrix $\sigma_2=\sigma_y$

\begin{align}
  \label{eq:2.11.11}
  Y=
  \begin{bmatrix}
    0 & -i \\
    i & 0
  \end{bmatrix}
\end{align}

\begin{align}
  \label{eq:2.11.12}
  \Rightarrow 0 = c(\lambda_y)&=|Y-\lambda_y I|
  =
  \begin{vmatrix}
    -\lambda_y & -i \\
    i & -\lambda_y
  \end{vmatrix}
  =\lambda_y^2-1
\end{align}

\begin{align}
  \label{eq:2.11.13}
  \therefore \lambda_y=\pm 1
\end{align}

If $\lambda_{y,1}=1$, its eigenvector $\ket{v_{y,1}}$ is:

\begin{align}
  \label{eq:2.11.14}
  \Rightarrow Y-\lambda_{y,1} I
  =
  \begin{bmatrix}
    -1 & -i \\
    i & -1
  \end{bmatrix} 
\end{align}

\begin{align}
  \label{eq:2.11.15}
  \Rightarrow
  \begin{bmatrix}
    -1 & -i \\
    i & -1
  \end{bmatrix}
  \begin{bmatrix}
    a_{y,1} \\
    a_{y,2}
  \end{bmatrix}
  =
  \begin{bmatrix}
    0 \\
    0
  \end{bmatrix}
\end{align}

\begin{align}
  \label{eq:2.11.16}
  \Rightarrow \ket{v_{y,1}} = \frac{1}{\sqrt{2}}
  \begin{bmatrix}
    1 \\
    i
  \end{bmatrix} 
\end{align}

If $\lambda_{y,2}=-1$, its eigenvector $\ket{v_{y,2}}$ is:

\begin{align}
  \label{eq:2.11.17}
  \Rightarrow Y-\lambda_{y,2} I
  =
  \begin{bmatrix}
    1 & -i \\
    i & 1
  \end{bmatrix} 
\end{align}

\begin{align}
  \label{eq:2.11.18}
  \Rightarrow
  \begin{bmatrix}
    1 & -i \\
    i & 1
  \end{bmatrix}
  \begin{bmatrix}
    b_{y,1} \\
    b_{y,2}
  \end{bmatrix}
  =
  \begin{bmatrix}
    0 \\
    0
  \end{bmatrix}
\end{align}

\begin{align}
  \label{eq:2.11.19}
  \Rightarrow \ket{v_{y,2}} = \frac{1}{\sqrt{2}}
  \begin{bmatrix}
    1 \\
    -i
  \end{bmatrix} 
\end{align}

The matrix representation in the basis of the eigenvectors for this operator, is just a diagonal matrix with entries equal to the eigenvalues:

\begin{align}
  \label{eq:2.11.20}
  \Rightarrow Y_{diag} =
  \begin{bmatrix}
    1 & 0 \\
    0 & -1
  \end{bmatrix} 
\end{align}


------------------------------------

Pauli matrix $\sigma_3=\sigma_z$

\begin{align}
  \label{eq:2.11.21}
  Z=
  \begin{bmatrix}
    1 & 0 \\
    0 & -1
  \end{bmatrix}
\end{align}

\begin{align}
  \label{eq:2.11.22}
  \Rightarrow 0 = c(\lambda_z)=|Z-\lambda_z I|
  &=
  \begin{vmatrix}
    1-\lambda_z & 0 \\
    0 & -1-\lambda_z
  \end{vmatrix}\\
  &=(1-\lambda_z)(-1-\lambda_z)
  =\lambda_z^2-1
\end{align}

\begin{align}
  \label{eq:2.11.23}
  \therefore \lambda_z=\pm 1
\end{align}

If $\lambda_{z,1}=1$, its eigenvector $\ket{v_{z,1}}$ is:

\begin{align}
  \label{eq:2.11.24}
  \Rightarrow Z-\lambda_{z,1} I
  =
  \begin{bmatrix}
    0 & 0 \\
    0 & -2
  \end{bmatrix} 
\end{align}

\begin{align}
  \label{eq:2.11.25}
  \Rightarrow
  \begin{bmatrix}
    0 & 0 \\
    0 & -2
  \end{bmatrix}
  \begin{bmatrix}
    a_{z,1} \\
    a_{z,2}
  \end{bmatrix}
  =
  \begin{bmatrix}
    0 \\
    0
  \end{bmatrix}
\end{align}

\begin{align}
  \label{eq:2.11.26}
  \Rightarrow \ket{v_{z,1}} = \frac{1}{\sqrt{2}}
  \begin{bmatrix}
    a_{z,1} \\
    0
  \end{bmatrix} 
\end{align}

If $\lambda_{z,2}=-1$, its eigenvector $\ket{v_{z,2}}$ is:

\begin{align}
  \label{eq:2.11.27}
  \Rightarrow Z-\lambda_{z,2} I
  =
  \begin{bmatrix}
    2 & 0 \\
    0 & 0
  \end{bmatrix} 
\end{align}

\begin{align}
  \label{eq:2.11.28}
  \Rightarrow
  \begin{bmatrix}
    2 & 0 \\
    0 & 0
  \end{bmatrix}
  \begin{bmatrix}
    b_{z,1} \\
    b_{z,2}
  \end{bmatrix}
  =
  \begin{bmatrix}
    0 \\
    0
  \end{bmatrix}
\end{align}

\begin{align}
  \label{eq:2.11.29}
  \Rightarrow \ket{v_{z,2}} = \frac{1}{\sqrt{2}}
  \begin{bmatrix}
    0 \\
    b_{z,2}
  \end{bmatrix} 
\end{align}

The matrix representation in the basis of the eigenvectors for this operator, is just a diagonal matrix with entries equal to the eigenvalues:

\begin{align}
  \label{eq:2.11.30}
  \Rightarrow Z_{diag} =
  \begin{bmatrix}
    1 & 0 \\
    0 & -1
  \end{bmatrix} 
\end{align}


\paragraph{2.12} %%%%%%%%%%%%%%%%%%%%%%%%%%%%%%%%%%%%%%%%%%%%











%%% Local Variables:
%%% mode: latex
%%% TeX-master: "chapter2_master"
%%% End:
