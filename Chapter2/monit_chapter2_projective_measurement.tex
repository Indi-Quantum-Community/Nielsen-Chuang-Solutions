\paragraph{2.58} \textbf{Average Observed value of $M$}%%%%%%%%%%%%%%%%%%%%%%%%%%%%%%%%%%%%%%%%%%%%
\\
$$ \langle M \rangle = \langle \psi |M| \psi \rangle = \langle \psi |m | \psi \rangle  = m \langle \psi | \psi \rangle = m$$

$$ \langle M^2 \rangle = \langle \psi |M^2| \psi \rangle = \langle \psi |m^2 | \psi \rangle  = m^2 \langle \psi | \psi \rangle = m^2$$

$$\text{ deviation } = \langle M^2 \rangle - \langle M \rangle^2 = m^2 - m^2 = 0$$

\paragraph{2.59} \textbf{Average and Standard Deviation of $X$}%%%%%%%%%%%%%%%%%%%%%%%%%%%%%%%%%%%%%%%%%%%%
\\
$$ \langle X \rangle = \langle 0 |X| 0\rangle = \langle 0 | 1 \rangle = 0$$

$$ \langle X^2 \rangle = \langle 0 |X^2 | 0 \rangle = \langle 0 | X | 1\rangle = \langle 0| 0 \rangle = 1$$
$$ \text{ standard deviation } = \sqrt{ \langle X^2 \rangle - \langle X \rangle^2 } = 1$$


\paragraph{2.60} \textbf{}%%%%%%%%%%%%%%%%%%%%%%%%%%%%%%%%%%%%%%%%%%%%
\\
\begin{align*}
    \vec{v} \cdot \vec{\sigma} &= \sum_{i=1}^3 v_i \sigma_i\\
    &= v_1 \begin{bmatrix}
        0 & 1 \\
        1 & 0
    \end{bmatrix}
    + v_2 \begin{bmatrix}
        0 & -i \\
        i & 0
    \end{bmatrix}
    + v_3 \begin{bmatrix}
        1 & 0 \\
        0 & -1
    \end{bmatrix} \\
    &= \begin{bmatrix}
        v_3 & v_1 - i v_2 \\
        v_1 + iv_2 & -v_3
    \end{bmatrix}
\end{align*}

\begin{align*}
    \det (\vec{v} \cdot \vec{\sigma}  - \lambda I) &= (v_3 - \lambda) (-v_3 - \lambda) - (v_1 - iv_2) (v_1 + iv_2)\\
    &= \lambda^2 - (v_1^2 + v_2^2  + v_3^2)\\
    &= \lambda^2 - 1 ~~~ (\because |\vec{v}| = 1)
\end{align*}
Eigenvalues are $\lambda = \pm 1$.


(i) if $\lambda = 1$
\begin{align*}
	\vec{v} \cdot \vec{\sigma}  - \lambda I &= \vec{v} \cdot \vec{\sigma}  - I\\
		&= \begin{bmatrix}
    		v_3 - 1 & v_1 - i v_2 \\
    		v_1 + i v_2 & - v_3 - 1
		\end{bmatrix}
\end{align*}

Normalized eigenvector is $\ket{\lambda_1} = \sqrt{ \frac{1+v_3}{2 }} \begin{bmatrix}
1 \\
\frac{1-v_3}{v_1 - iv_2}
\end{bmatrix} $.

$$
	\kb{\lambda_1} &= \frac{1+v_3}{2 } \begin{bmatrix}
		1 \\
		\frac{1-v_3}{v_1 - iv_2}
	\end{bmatrix}
	\begin{bmatrix}
   		1 &
   		\frac{1-v_3}{v_1 + iv_2}
	\end{bmatrix}\\
%
	&=
	 \frac{1+v_3}{2 } \begin{bmatrix}
    	 1 & \frac{v_1 - iv_2}{1 + v_3} \\
    	 \frac{v_1 + iv_2}{1 + v_3} & \frac{1-v_3}{1+v_3}
	 \end{bmatrix} \\ $$
  $$
	 &=
	 \frac{1}{2} \begin{bmatrix}
    	 1+v_3 & v_1 - iv_2 \\
    	 v_1 + iv_2 & 1 - v_3
	 \end{bmatrix} \\
	 &=
	  \frac{1}{2} \left( I + \begin{bmatrix}
    	 v_3 & v_1 - iv_2 \\
    	 v_1 + iv_2 & - v_3
	 \end{bmatrix} \right) \\
	 &=
	 \frac{1}{2} (I + \vec{v} \cdot \vec{\sigma} )
$$



(ii) If $\lambda = -1$.
\begin{align*}
	\vec{v} \cdot \vec{\sigma}  - \lambda I &= \vec{v} \cdot \vec{\sigma}  + I\\
	&= \begin{bmatrix}
		v_3 + 1 & v_1 - i v_2 \\
		v_1 + i v_2 & - v_3 + 1
	\end{bmatrix}
\end{align*}

Normalized eigenvalue is $\ket{\lambda_{-1}} = \sqrt{ \frac{1-v_3}{2 }} \begin{bmatrix}
    1 \\
    - \frac{1+v_3}{v_1 - iv_2}
\end{bmatrix} $.



	$$ \ket{\lambda_{-1}} &= \frac{1 - v_3}{2} 
 \begin{bmatrix}
	1 \\
	- \frac{1+v_3}{v_1 - iv_2}
	\end{bmatrix}
	\begin{bmatrix}
		1 & - \frac{1+v_3}{v_1 + iv_2}
	\end{bmatrix}\\
	&=
	\frac{1 - v_3}{2} \begin{bmatrix}
		1 & - \frac{v_1 - iv_2}{1 - v_3} \\
		- \frac{v_1 + iv_2}{1 - v_3} & \frac{1+v_3}{1 - v_3}
	\end{bmatrix} \\ $$
 $$
	&=
	\frac{1}{2} \begin{bmatrix}
		1 - v_3 & -(v_1 - iv_2) \\
		- (v_1 + iv_2) & 1 + v_3
	\end{bmatrix} \\
	&=
	\frac{1}{2} \left( I - \begin{bmatrix}
		v_3 & v_1 - iv_2 \\
		(v_1 + iv_2 & - v_3
	\end{bmatrix} \right)\\
	&= \frac{1}{2} (I - \vec{v} \cdot \vec{\sigma} ).
$$


	This proof has a defect.
	The case $(v_1,v_2,v_3) = (0,0,1)$, second component of eigenstate, $\frac{1-v_3}{v_1 - iv_2}$, diverges.
	So I implicitly assume $v_1 - iv_2 \neq 0$. Hence my proof is incomplete.

	Since the exercise doesn't require explicit form of projector, we should prove the problem more abstractly.
	In order to prove, we use the following properties of $\vec{v} \cdot \vec{\sigma}$
	\begin{itemize}
		\item $\vec{v} \cdot \vec{\sigma}$ is Hermitian
		\item $(\vec{v} \cdot \vec{\sigma})^2 = I$ where $\vec{v}$ is a real unit vector.
	\end{itemize}

	We can easily check above conditions.
	\begin{align*}
	(\vec{v} \cdot \vec{\sigma})^\dagger &= (v_1 \sigma_1 + v_2 \sigma_2 + v_3 \sigma_3)^\dagger\\
	&= v_1 \sigma_1^\dagger + v_2 \sigma_2^\dagger + v_3 \sigma_3^\dagger\\
	&= v_1 \sigma_1 + v_2 \sigma_2 + v_3 \sigma_3~~~(\because \text{Pauli matrices are Hermitian.})\\
	&= \vec{v} \cdot \vec{\sigma}
	\end{align*}

	\begin{align*}
	(\vec{v} \cdot \vec{\sigma})^2 &= \sum_{j,k=1}^3 (v_j \sigma_j)  (v_k \sigma_k)\\
	&= \sum_{j,k=1}^3 v_j v_k \sigma_j \sigma_k\\
	&= \sum_{j,k=1}^3 v_j v_k \left(\delta_{jk}I + i \sum_{l=1}^3 \epsilon_{jkl}\sigma_l \right) ~~~(\because \text{eqn}(2.78)~ \text{page} 78)\\
	&= \sum_{j,k=1}^3 v_j v_k \delta_{jk}I  + i \sum_{j,k,l=1}^3 \epsilon_{jkl} v_j v_k \sigma_l\\
	&= \sum_{j=1}^3 v_j^2 I\\
	&= I ~~~\left(\because \sum_j v_j^2 = 1 \right)
	\end{align*}



		Suppose $\ket{\lambda}$ is an eigenstate of $\vec{v} \cdot \vec{\sigma}$ with eigenvalue $\lambda$. Then
		\begin{align*}
		\vec{v} \cdot \vec{\sigma} \ket{\lambda} = \lambda \ket{\lambda}\\
		(\vec{v} \cdot \vec{\sigma})^2 \ket{\lambda} = \lambda^2 \ket{\lambda}
		\end{align*}
		On the other hand $(\vec{v} \cdot \vec{\sigma})^2 = I$,
		\begin{align*}
		(\vec{v} \cdot \vec{\sigma})^2 \ket{\lambda} = I \ket{\lambda} = \ket{\lambda}\\
		\therefore \lambda^2\ket{\lambda} = \ket{\lambda}.
		\end{align*}
		Thus $\lambda^2 = 1 \Rightarrow \lambda = \pm 1$. Therefore $\vec{v} \cdot \vec{\sigma}$ has eigenvalues $\pm 1$.

		Let $\ket{\lambda_1}$ and $\ket{\lambda_{-1}}$ are eigenvectors with eigenvalues $1$ and $-1$, respectively.
		I will prove that $P_{\pm} = \ket{\lambda_{\pm 1}}$.

		In order to prove above equation, all we have to do is prove following condition. 

			
				$$\braket{\psi | (P_\pm - \kb{\lambda_{\pm 1}})| \psi} = 0 \text{ for all } \ket{\psi} \in \mathds{C}^2$$
			


		Since $\vec{v} \cdot \vec{\sigma}$ is Hermitian, $\ket{\lambda_1}$ and $\ket{\lambda_{-1}}$ are orthonormal vector ($\because $ Exercise 2.22).
		Let $\ket{\psi} \in \mathds{C}^2$ be an arbitrary state. $\ket{\psi}$ can be written as
		\begin{align*}
		\ket{\psi} = \alpha \ket{\lambda_1} + \beta \ket{\lambda_{\pm 1}} ~~(|\alpha|^2 + |\beta|^2 = 1, \alpha, \beta \in \mathds{C}).
		\end{align*}

		\begin{align*}
		\braket{\psi | (P_{\pm} - \kb{\lambda_\pm})| \psi}
		%		&= \braket{\psi | \left(\frac{1}{2} (I \pm \vec{v} \cdot \vec{\sigma}) - \kb{\lambda_\pm}\right)| \psi}\\
		&= \braket{\psi | P_\pm | \psi} - \braket{\psi | \lambda_\pm} \braket{\lambda_\pm | \psi}.\\
		\braket{\psi | P_\pm | \psi} &= \braket{\psi | \frac{1}{2}(I \pm \vec{v} \cdot \vec{\sigma}) | \psi}\\
		&= \frac{1}{2} \pm \frac{1}{2} \braket{\psi | \vec{v} \cdot \vec{\sigma})| \psi}\\
		&= \frac{1}{2} \pm \frac{1}{2} (|\alpha|^2 - |\beta|^2)\\
		&= \frac{1}{2} \pm \frac{1}{2} (2|\alpha|^2 - 1) ~~~(\because |\alpha|^2 + |\beta|^2 = 1)\\
		\braket{\psi | \lambda_1} \braket{\lambda_1 | \psi} &= |\alpha|^2\\
		\braket{\psi | \lambda_{-1}} \braket{\lambda_{-1}| \psi} &= |\beta|^2 = 1  - |\alpha|^2
		\end{align*}

		Therefore $\braket{\psi | (P_\pm - \kb{\lambda_{\pm 1}})| \psi} = 0$ for all $\ket{\psi} \in \mathds{C}^2$.
		Thus $P_\pm = \kb{\lambda_{\pm 1}}$.






\paragraph{2.61} \textbf{Calculate the Probability}%%%%%%%%%%%%%%%%%%%%%%%%%%%%%%%%%%%%%%%%%%%%
\\

$$ \langle \lambda_1 | 0 \rangle \langle 0|\lambda_1 \rangle - \langle 0 | \lambda_1 \rangle \langle \lambda_1|0\rangle$$
$$ = \langle 0 | \frac{1}{2}(I + \vec{v}\cdot \vec{\sigma}|0\rangle$$
$$ \frac{1}{2}(1+ v_3)$$

and the state after measurement is 
$$ \frac{|\lambda_1\rangle \langle \lambda_1 | 0\rangle}{\sqrt{\langle 0 | \lambda_1 \rangle \langle \lambda_1| 0\rangle}} = \frac{1}{\sqrt{\frac{1}{2} ( 1 + v_3)}} \cdot \frac{1}{2}\begin{bmatrix}
    1 + v_3 \\ v_1 + \iota v_2
\end{bmatrix}$$

$$= \sqrt{\frac{1}{2}( 1 + v_3)} \cdot \begin{bmatrix}
    1  \\ \frac{v_1 + \iota v_2}{1 + v_3}
\end{bmatrix}$$

$$ = \sqrt{\frac{1+ v_3}{2}} \begin{bmatrix}
    1 \\ \frac{1 - v_3}{v_1 - \iota v_2}
\end{bmatrix}$$

$$ = |\lambda_1\rangle$$
