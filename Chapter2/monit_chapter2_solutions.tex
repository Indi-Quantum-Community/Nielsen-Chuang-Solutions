



\paragraph{2.19} \textbf{Pauli Matrices are unitary and Hermitian} \\%%%%%%%%%%%%%%%%%%%%%%%%%%%%%%%%%%%%%%%%%%%%
We know the first Pauli matrix is the identity matrix, which is by definition unitary and hermitian:

\begin{align}
  \label{eq:2.19.1}
  \sigma_0^{\dagger} = 
  \begin{bmatrix}
    1 & 0 \\
    0 & 1
  \end{bmatrix} = \sigma_0
\end{align} 


So, we have :
\begin{align}
    \sigma_0^{\dagger} \sigma_0 = 
    \begin{bmatrix}
        1 & 0 \\
        0 & 1
    \end{bmatrix}
    \begin{bmatrix}
        1 & 0 \\
        0 & 1
    \end{bmatrix}
    =
    \begin{bmatrix}
        1 & 0 \\
        0 & 1
    \end{bmatrix} = I
\end{align}

Similarly for the other Pauli Matrices, we have


\begin{align}
  \label{eq:2.19.2}
  \sigma_1^{\dagger} = 
  \begin{bmatrix}
    0 & 1 \\
    1 & 0
  \end{bmatrix} = \sigma_1
\end{align} 


So, we have :
\begin{align}
    \sigma_1^{\dagger} \sigma_1 = 
    \begin{bmatrix}
        0 & 1 \\
        1 & 0
    \end{bmatrix}
    \begin{bmatrix}
        0 & 1 \\
        1 & 0
    \end{bmatrix}
    =
    \begin{bmatrix}
        1 & 0 \\
        0 & 1
    \end{bmatrix} = I
\end{align}

For $\sigma_2$ , we have:

\begin{align}
  \label{eq:2.19.2}
  \sigma_2^{\dagger} = 
  \begin{bmatrix}
    0 & -i \\
    i & 0
  \end{bmatrix} = \sigma_2
\end{align} 


So, we have :
\begin{align}
    \sigma_2^{\dagger} \sigma_2 = 
    \begin{bmatrix}
        0 & -i \\
        i & 0
    \end{bmatrix}
    \begin{bmatrix}
        0 & -i \\
        i & 0
    \end{bmatrix}
    =
    \begin{bmatrix}
        1 & 0 \\
        0 & 1
    \end{bmatrix} = I
\end{align}

For $\sigma_3$:

\begin{align}
  \label{eq:2.19.3}
  \sigma_3^{\dagger} = 
  \begin{bmatrix}
    1 & 0 \\
    0 & -1
  \end{bmatrix} = \sigma_3
\end{align} 


So, we have :
\begin{align}
    \sigma_3^{\dagger} \sigma_3 = 
    \begin{bmatrix}
        1 & 0 \\
        0 & -1
    \end{bmatrix}
    \begin{bmatrix}
        1 & 0 \\
        0 & -1
    \end{bmatrix}
    =
    \begin{bmatrix}
        1 & 0 \\
        0 & 1
    \end{bmatrix} = I
\end{align}


Hence, all Pauli matrices are unitary and hermitian.










\paragraph{2.20} \textbf{Basis Change} %%%%%%%%%%%%%%%%%%%%%%%%%%%%%%%%%%%%%%%%%%%%

Given operator $A$, with two matrix representations $A'$ and $A''$, on a vector space $V$ with two different orthonormal basis, $|v_i\rangle$ and $|w_i\rangle$.
The relation between them:

$$ A'_{ij} = \langle v_i |A | v_j \rangle $$
$$  \implies \sum_k \langle v_i | w_k \rangle \langle w_k |A | v_j \rangle $$
$$  \implies \sum_{k,l} \langle v_i | w_k \rangle \langle w_k |A | w_l \rangle \langle w_l | v_j \rangle $$
$$  \implies \sum_{k,l} \langle v_i |U| v_k \rangle \langle w_k |A | w_l \rangle \langle v_l |U^{\dagger}| v_j \rangle $$
$$  \implies \sum_{k,l} U_{ik} A_{kl}^{''} U_{lj}^{\dagger} $$

where $U \equiv \sum_m |w_m\rangle \langle v_m|$

\paragraph{2.21} %%%%%%%%%%%%%%%%%%%%%%%%%%%%%%%%%%%%%%%%%%%%

