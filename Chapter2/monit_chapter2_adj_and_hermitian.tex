



\paragraph{2.19} \textbf{Pauli Matrices are unitary and Hermitian} \\%%%%%%%%%%%%%%%%%%%%%%%%%%%%%%%%%%%%%%%%%%%%
We know the first Pauli matrix is the identity matrix, which is by definition unitary and hermitian:

\begin{align}
  \label{eq:2.19.1}
  \sigma_0^{\dagger} = 
  \begin{bmatrix}
    1 & 0 \\
    0 & 1
  \end{bmatrix} = \sigma_0
\end{align} 


So, we have :
\begin{align}
    \sigma_0^{\dagger} \sigma_0 = 
    \begin{bmatrix}
        1 & 0 \\
        0 & 1
    \end{bmatrix}
    \begin{bmatrix}
        1 & 0 \\
        0 & 1
    \end{bmatrix}
    =
    \begin{bmatrix}
        1 & 0 \\
        0 & 1
    \end{bmatrix} = I
\end{align}

Similarly for the other Pauli Matrices, we have


\begin{align}
  \label{eq:2.19.2}
  \sigma_1^{\dagger} = 
  \begin{bmatrix}
    0 & 1 \\
    1 & 0
  \end{bmatrix} = \sigma_1
\end{align} 


So, we have :
\begin{align}
    \sigma_1^{\dagger} \sigma_1 = 
    \begin{bmatrix}
        0 & 1 \\
        1 & 0
    \end{bmatrix}
    \begin{bmatrix}
        0 & 1 \\
        1 & 0
    \end{bmatrix}
    =
    \begin{bmatrix}
        1 & 0 \\
        0 & 1
    \end{bmatrix} = I
\end{align}

For $\sigma_2$ , we have:

\begin{align}
  \label{eq:2.19.2}
  \sigma_2^{\dagger} = 
  \begin{bmatrix}
    0 & -i \\
    i & 0
  \end{bmatrix} = \sigma_2
\end{align} 


So, we have :
\begin{align}
    \sigma_2^{\dagger} \sigma_2 = 
    \begin{bmatrix}
        0 & -i \\
        i & 0
    \end{bmatrix}
    \begin{bmatrix}
        0 & -i \\
        i & 0
    \end{bmatrix}
    =
    \begin{bmatrix}
        1 & 0 \\
        0 & 1
    \end{bmatrix} = I
\end{align}

For $\sigma_3$:

\begin{align}
  \label{eq:2.19.3}
  \sigma_3^{\dagger} = 
  \begin{bmatrix}
    1 & 0 \\
    0 & -1
  \end{bmatrix} = \sigma_3
\end{align} 


So, we have :
\begin{align}
    \sigma_3^{\dagger} \sigma_3 = 
    \begin{bmatrix}
        1 & 0 \\
        0 & -1
    \end{bmatrix}
    \begin{bmatrix}
        1 & 0 \\
        0 & -1
    \end{bmatrix}
    =
    \begin{bmatrix}
        1 & 0 \\
        0 & 1
    \end{bmatrix} = I
\end{align}


Hence, all Pauli matrices are unitary and hermitian.










\paragraph{2.20} \textbf{Basis Change} %%%%%%%%%%%%%%%%%%%%%%%%%%%%%%%%%%%%%%%%%%%%

Given operator $A$, with two matrix representations $A'$ and $A''$, on a vector space $V$ with two different orthonormal basis, $|v_i\rangle$ and $|w_i\rangle$.
The relation between them:

$$ A'_{ij} = \langle v_i |A | v_j \rangle $$
$$  \implies \sum_k \langle v_i | w_k \rangle \langle w_k |A | v_j \rangle $$
$$  \implies \sum_{k,l} \langle v_i | w_k \rangle \langle w_k |A | w_l \rangle \langle w_l | v_j \rangle $$
$$  \implies \sum_{k,l} \langle v_i |U| v_k \rangle \langle w_k |A | w_l \rangle \langle v_l |U^{\dagger}| v_j \rangle $$
$$  \implies \sum_{k,l} U_{ik} A_{kl}^{''} U_{lj}^{\dagger} $$

where $U \equiv \sum_m |w_m\rangle \langle v_m|$

\paragraph{2.21} \textbf{Spectral Decomposition}%%%%%%%%%%%%%%%%%%%%%%%%%%%%%%%%%%%%%%%%%%%%
\\

The \emph{spectral decomposition} is an extremely useful representation theorem for normal operators. It states:
Any normal operator $M$ on a vector space $V$ is diagonal with respect to some orthonormal basis for $V$. Conversely, any diagonalizable operator is normal.

We will solve this my the method of induction. We'll assume the $n = 1$ case is trivial. Now, let $\lambda $ be an eigenvalue of $M$, $P$ is the projector on to the $\lambda$ eigenspace, and $Q$ the projector onto the orthogonal component. 
Then, 
$$ M = (P+Q) M (P+Q) $$
$$ M = PMP + QMP + PMQ + QMQ $$
Now, obviously $PMP = \lambda P , QMP = 0$, since $M$ takes the $P$ subspace onto itself. 

Furthermore, we can take the Hermitian conjugate of both sides to show that:
$$ 0 = (QMP)^{\dagger} = PMQ$$

Showing that $QMQ$ is normal is trivial, since $QMQ = (QMQ)^{\dagger}$. 
Since $PMP$ is diagonal with respect to some vector space, and $QMQ$ is normal , and thus diagonal with respect to another orthogonal vector space, $PMP + QMQ$ is diagonal with respect to some orthonormal basis for the total vector space.

\\

Alternatively

\\

Suppose $M$ be hermitian. Then $M = M^{\dagger}$
$$ M = IMI$$
$$ M + (P+Q)M(P+Q)$$
$$ M = PMP + QMP +PMQ + QMQ$$

Now, $PMP = \lambda P$, $QMP = 0 $, $PMQ = PM^{\dagger}Q = (QMP)^{*} = 0 $. Thus $M = PMP + QMQ$

Now, we need to prove $QMQ $ is normal.

$$ QMQ(QMQ)^{\dagger} = QMQQM^{\dagger}Q$$
$$ \implies QM^{\dagger}QQMQ$$
$$\implies (QM^{\dagger}Q)QMQ$$

Therefore $QMQ$ is normal. By induction, $QMQ$ is diagonal.

\paragraph{2.22} \textbf{Two Eigenvectors of a Hermitian operator with different eigenvalues are necessarily orthogonal}%%%%%%%%%%%%%%%%%%%%%%%%%%%%%%%%%%%%%%%%%%%%
\\

Let us assume $M $ be the hermitian operator and $|v_i\rangle$ are the eigenvectors of $M$ with eigenvalues $\lambda_i$. Then
$$ \langle v_i |M |v_j \rangle = \lambda_j \langle v_i | v_j\rangle$$
, Similarly
$$\langle v_i|M|v_j\rangle = \langle v_i | M^{\dagger} | v_j \rangle = \langle v_j | M | v_i \rangle^{*} =  \lambda_i^{*}\langle v_j | v_i \rangle^{*} = \lambda_i^{*}\langle v_i | v_j \rangle = \lambda_i\langle v_j | v_i \rangle$$

Thus,
$$ (\lambda_i - \lambda_j) \langle v_i | v_j \rangle = 0$$

If $\lambda_i \ne \lambda_j$, then  $\langle v_i | v_j\rangle = 0$

\paragraph{2.23} \textbf{Show that the eigenvalues of a projector $P$ are all either $0$ or $1$}%%%%%%%%%%%%%%%%%%%%%%%%%%%%%%%%%%%%%%%%%%%%
\\

Let us assume $P$ is the projector and $|\lambda\rangle$ are the eigenvectors of the projector, with eigen values $\lambda$. Since its a projector operator $P^2 = P$

$$ P |\lambda\rangle = \lambda |\lambda\rangle \text{ and } P|\lambda\rangle = P^2|\lambda\rangle = \lambda P |\lambda\rangle = \lambda^2 |\lambda\rangle$$

Therefore,
$$ \lambda = \lambda^2$$
$$ \lambda(\lambda-1) = 0$$
$$\lambda = 0 \text{ or } 1$$

\paragraph{2.24} \textbf{Hermiticity of positive operators}%%%%%%%%%%%%%%%%%%%%%%%%%%%%%%%%%%%%%%%%%%%%
\\
Suppose $A$ be a positive operator, $A$ can be decomposed as:
$$ A = \frac{A + A^{\dagger}}{2} + \iota \frac{A - A^{\dagger}}{2i}$$
$$ = B + \iota C \textbf{ where } B = \frac{A + A^{\dagger}}{2}, C = \frac{A - A^{\dagger}}{2i}$$

Now the operators $B$ and $C$ are hermitian.

$$ \langle v | A | v\rangle = \langle v |B = \iota C| v \rangle$$
$$ \implies = \langle v |B | v \rangle + \iota \langle v |C|v\rangle$$
$$ \implies = \alpha + \iota \beta \text{ where } \alpha = \langle v |B | v\rangle, \beta = \langle v | C | v\rangle$$

Since $B$ and $C$ are hermitian, $\alpha, \beta \in \mathbb{R}$. From definition of positive operator , $\beta$ should be vanished because $\langle v | A | v\rangle $ is real. Hence $\beta = \langle v |C|\rangle = 0$ for all $|v\rangle$, i.e $C=0$.

Therefore $A = A^{\dagger}$.


\paragraph{2.25} \textbf{For any operator $A$, $A^{\dagger}A$ is positive}%%%%%%%%%%%%%%%%%%%%%%%%%%%%%%%%%%%%%%%%%%%%
\\

$$\langle \psi | A^{\dagger } A |\psi\rangle = || A |\psi\rangle ||^2 \ge 0 \forall |\psi\rangle$$

Thus $A^{\dagger} A $ is positive.

